%\documentclass{winnower}
\documentclass{article}
\begin{document}

\title{Generative Adversarial Networks}

\author{Cameron Fabbri}
%\affil{Computer Science and Engineering, University of Minnesota}
\date{4/24/2017}

\maketitle

\begin{abstract}
We provide an overview of Generative Adversarial Networks as a class of generative models for
image generation, and discuss how they can be applied towards the task of automatic image colorization. 
We first give a brief overview of Deep Learning and the recent advances in generative models.
We then discuss several varients of Generative Adversarial Networks, the problems they pose during training,
and theoretical methods towards stabalizing training.
\end{abstract}


%-------------------------------------------------%
\section{Introduction}
%-------------------------------------------------%
Deep learning has recently shown impressive results towards various problems in multiple domains such as speech recognition,
image classification, image segmentation, and reinforcement learning []. Until recently, much of the focus was towards
discriminative models, which aims to map a high-dimensional input, such as an image, to a class label. Deep generative models,
such as Deep Botlzmann Machines and Deep Belief Networks, have not had the same level of success.
Generative Adversarial Networks (GANs)[1] are a class of generative models that have shown great success in generatig realistic
images. Despite their success, they are known to be very difficult to train, and are extremely sensitive to modifications.
For this reason it is not yet straightforward to directly apply them towards a different problem.

Since their introduction in 2014, there have been several large contributions made towards stabalizing and understanding
the training process of GANs. We discuss and compare four different objective functions used for training GANs:

\begin{itemize}
   \item Classification loss
   \item Least Squares loss
   \item Energy loss
   \item Wasserstein
\end{itemize}

\noindent We provide a background on deep learning in Section 2, and in Section 3 discuss the varients of GANs mentioned above.
Section 4 demonstrates these varients on the task of automatically colorizing grayscale images. The Appendix provides further
results from our experiments.

%-------------------------------------------------%
\section{Background}
%-------------------------------------------------%
This section provides a brief overview of deep learning (Goodfellow, et al., 2016).

%-------------------------------------------------%
\subsection{Deep Learning}
%-------------------------------------------------%
Deep learning is a class of machine learning algorithms that use one or more \textit{hidden layers} between the input and output
to automatically learn a heirarchy of concepts.

%-------------------------------------------------%
\subsection{Convolutional Neural Networks}
%-------------------------------------------------%


%-------------------------------------------------%
\subsection{Generative Models}
%-------------------------------------------------%
Much of the early success in deep learning was geared towards discriminative models.



%-------------------------------------------------%
\section{Generative Adversarial Networks}
%-------------------------------------------------%


%-------------------------------------------------%
\subsection{Deep Convolutional GANs}
%-------------------------------------------------%


%-------------------------------------------------%
\subsection{Conditional Generative Adversarial Networks}
%-------------------------------------------------%


%-------------------------------------------------%
\subsection{Least Squares GANs}
%-------------------------------------------------%


%-------------------------------------------------%
\subsection{Energy-Based GANs}
%-------------------------------------------------%	 


%-------------------------------------------------%
\subsection{Wasserstein GANs}
%-------------------------------------------------%


%-------------------------------------------------%
\section{Colorization}
%-------------------------------------------------%
We now show how adversarial networks can be used for generating a plausible color version of a grayscale image.
The problem is set up as a cGAN, where the generator and descriminator are both conditioned on the grayscale image.






\bibliographystyle{abbrvnat}
[1] Graves, Alex, and Navdeep Jaitly. "Towards End-To-End Speech Recognition with Recurrent Neural Networks." ICML. Vol. 14. 2014.
[2] Goodfellow, Ian, Yoshua Bengio, and Aaron Courville. Deep learning. MIT Press, 2016.
\bibliography{winnower_template}

\appendix

\section{Appendix}
Here we show 


\end{document}
