\documentclass{article}

\begin{document}

\centerline{\sc \large Project Proposal}
\vspace{.5pc}
\centerline{\sc Team Name: Just Add Another Layer}
\centerline{\sc Team Name: Did You Try Adding Another Layer?}
\centerline{\sc Team Name: }
\centerline{\sc Team Members: Cameron Fabbri, Jahidul Islam}
\centerline{\sc 2/5/2017}
\vspace{2pc}

\centerline{\sc \large Generative Adversarial Networks for Automatic Image Colorization }

\section{Introduction}
There exists a large amount of photographs and videos, mainly antique, that lack color.
Providing color to these images provides a modern view to these images. For a human,
the task of colorizing these black and white photos leaves open room for imagination. While
some objects commonly hold the same color (i.e the sky is \textit{usually} blue), many are
left to the imagination. For example, given a black and white photo of someone wearing a dark
colored shirt, it would be very difficult or near impossible to tell whether that shirt was
blue or green. Therefore, given there is not one correct answer, automatic colorization is an
ill-posed problem.

\section{Approach}
Recent advances in deep learning and big data provide us with a good starting point for tackling this
problem. We propose to leverage two specific areas to solve this problem. The first is deep Convolutional
Neural Networks. Literature has shown[1] that these can be used to provide a \textit{plausible}
colorization of a black and white photo. The second is Generative Adversarial Networks (GANs). Recently,
GANs have shown very promising results for generating data. We believe the combination of these two
methods should provide better results for automatic colorization.

\section{Objectives}
Given the nature of current research in deep learning, there are many variations in which we
can try. 
While much of the work will overlap as the project progresses, our current plan for task
delegation is as follows.

Cameron: 1, 3, etc
Jahidul: 2, 4, etc

\section{Timeframe}
Goals to achieve by certain dates.


\section{Evaluations}
How we are going to evaluate the results we get.



\section{References}
[1] https://arxiv.org/pdf/1603.08511.pdf





\end{document}
