\documentclass[11pt]{article}
\usepackage{times}
\usepackage{graphicx}
\usepackage{amsmath}

\begin{document}

\centerline{\sc \large Project Progress Report}
\vspace{.5pc}
\centerline{\sc Team Name: Yet Another Layer [YAL]}
\centerline{\sc Team Members: Cameron Fabbri, MD Jahidul Islam}
\centerline{\sc 3/27/2017}
\vspace{2pc}

\centerline{\sc \large Generative Adversarial Networks for Automatic Image Colorization }

\section{Objectives}
Our objectives given in the project proposal are stated below. 

\noindent (1) Implement the state of the art[1] for image colorization. \newline
\noindent (2) Implement Deep Convolutional Generative Adversarial Networks (DCGANs)[3]. \newline
\noindent (3) Implement Energy-Based Generative Adversarial Networks (EBGANs).[2] \newline
\noindent (4) Explore methods to pretrain the model used in 1 and fine tune it as a generator in 2 and 3.
\newline
\noindent (5) Time permitting, develop our own GAN architecture for comparison. \newline

\noindent Jahidul: 1, 4, 5 \newline
\noindent Cameron: 2, 3, 5 \newline

\noindent Cameron has implemented an adversarial network similar to the architecture used in [6]. Our model
is capable of using a combination of multiple loss functions. These include the typical L1 and L2 losses,
as well as various types of adversarial losses, such as the Wasserstein GAN [7] and the Least Squares GAN
[8]. 
\vspace{1pc}

\noindent Jahidul has implemented the Colorful Colorization network shown in [1] using L2 and L1 loss function. The original paper also implemented their own loss function, which is a major contribution to their work. Jahidul is currently implementing that loss function. We expect an improved colorization performance of the model using their customized loss function. 
\vspace{3mm}

\noindent Both of these have been trained and tested on the CelebA dataset [5]. Samples are shown below.

\begin{figure*}[t]
\vspace{-10mm}
\hspace{-25mm}
\includegraphics [scale=0.32]{1.pdf}
\vspace{-26mm}
\caption{Results for colorization with different models}
%\vspace{-5mm}
\label{fig:1}
\end{figure*}

\begin{figure*}[t]
\vspace{-10mm}
\hspace{-20mm}
\includegraphics [scale=0.34]{2.pdf}
\vspace{-16mm}
\caption{Results for colorization with different models (contd.)}
%\vspace{-5mm}
\label{fig:2}
\end{figure*}

\begin{figure*}[t]
\vspace{-10mm}
\hspace{-20mm}
\includegraphics [scale=0.34]{3.pdf}
\vspace{-16mm}
\caption{Results for colorization with different models (contd.)}
%\vspace{-5mm}
\label{fig:2}
\end{figure*}

\begin{figure*}[t]
\vspace{-10mm}
\hspace{-20mm}
\includegraphics [scale=0.35]{4.pdf}
\vspace{-18mm}
\caption{Results for colorization with different models (contd.)}
%\vspace{-5mm}
\label{fig:2}
\end{figure*}

\newpage
\section{To Do}
We still have a number of things to test, such as pretraining the generator, implementing Energy-Based GANs,
combining the loss function in [1] with GAN loss, and training on multiple classes. While some of our results
show good performance on one class, we are uncertain how this will scale to multiple classes.



\section{References}
[1] Zhang, Richard, Phillip Isola, and Alexei A. Efros. "Colorful image colorization." 
European Conference on Computer Vision. Springer International Publishing, 2016.
\vspace{2pt}

\noindent [2] J. Zhao, M. Mathieu, and Y. LeCun.  Energy-based Generative Adversarial 
Network. ArXiv e-prints, September 2016.
\vspace{2pt}

\noindent [3] Radford, Alec, Luke Metz, and Soumith Chintala. "Unsupervised representation learning with deep
convolutional generative adversarial networks." arXiv preprint arXiv:1511.06434 (2015).

\noindent [4] Goodfellow, Ian, et al. "Generative adversarial nets." Advances in neural information processing systems. 2014.

\noindent [5] Liu, Ziwei, et al. "Deep learning face attributes in the wild." Proceedings of the IEEE International Conference on Computer Vision. 2015.

\noindent [6] Isola, Phillip, et al. "Image-to-image translation with conditional adversarial networks." arXiv preprint arXiv:1611.07004 (2016).

\noindent [7] Arjovsky, Martin, Soumith Chintala, and Léon Bottou. "Wasserstein gan." arXiv preprint arXiv:1701.07875 (2017).

\noindent [8] Mao, Xudong, et al. "Least Squares Generative Adversarial Networks." arXiv preprint arXiv:1611.04076 (2016).

\end{document}
